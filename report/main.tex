\documentclass[
	10pt,
	parskip=half-,	
	paper=a4,
	english
	]{scrartcl}	
\usepackage{graphicx} % Required for inserting images
\usepackage{datetime}

\usepackage[english]{babel} 
\textheight = 220mm		
\footskip = 2cm		
\title{Title of the project}
\author{Author}
\date{\newdateformat{monthyeardate}{%
  \monthname[\THEMONTH], \THEYEAR}}

\usepackage{algorithm}
\usepackage{algpseudocode}
\usepackage{amsmath}
\usepackage{amssymb}
\usepackage{hyperref}
\hypersetup{
    colorlinks=true,
    linkcolor=black,
    filecolor=magenta,      
    urlcolor=blue,
    pdftitle={projectname},
}
\usepackage{fancyhdr}
\usepackage{booktabs}
\pagestyle{fancy}
\lhead{AC via maximum incremental path integral}
\rhead{Bernat Comas}
\date{}

\begin{document}

\begin{titlepage}
    \centering
    \vspace*{\fill}

    {\Huge\bfseries Agglomerative clustering via maximum incremental path integral: an implementation}
    \vspace{1cm}
    
    {\bfseries Unsupervised and Reinforcement Learning\\} % Second line of the title
    \vspace{0.5cm}
    {\bfseries \today}
    \vspace{1.5cm}

    {\Large Bernat Comas i Machuca}
    \vspace{0.5cm}

    {\Large \date{\newdateformat{monthyeardate}{%
  \monthname[\THEMONTH], \THEYEAR}}}

    \vspace*{\fill} % Vertically center the content

\end{titlepage}

\newpage			
\tableofcontents
\newpage
\section {Introduction to the problem}

The paper Agglomerative clustering via maximum incremental path integral (2013) \cite{citation1} presents a new agglomerative clustering algorithm that computes cluster similarity based on path graphs that capture the data structure. To make this computation each cluster is treated as a dynamic system and the algorithm measures stability using the Path Integral, a concept extracted from statistical mechanics and quantum mechanics. The algorithm decides how to merge clusters based on how much their stability changes, which results in an efficient solution with linear time complexity.

To define the structural descriptors that allow to compute this concept of similarity, a neighborhood graph is created. This allows it to avoid relying as much in pairwise distances (only used for graph initialization), and instead define the affinity between clusters by how much their stability changes when merged, which is measured through the path integral.

This has several advantages:

\begin{itemize}
    \item Works well for data lying on low-dimensional manifolds.
    \item It does not rely on approximation or eigen-decomposition, making it more robust to noise.
    \item It does not make assumptions on data distribution, offering better generalization.
    \item Performs well on multi-scale data
    \item It is efficiently computed by the presented algorithm in linear time complexity
\end{itemize}

Over the course of this project, we are going to implement the Path Integral Clustering algorithm in an efficient way, and then we are going to compare it to other well-known clustering algorithms. In this section we are going to describe the algorithm, section 2 contains a description of the implementation , in section 3 the algorithm experiments and comparison is performed, and finally the last section is devoted to describing the conclusions of the project.

\subsection{Cluster initialization}

Because we are working with an Agglomerative Clustering algorithm, we have to define how will the initial clusters be formed. We could initialize the samples each in its own cluster and start the iteration, but instead the paper proposes to use nearest neighbor merging. 

The algorithm of nearest neighbor merging consists of having each sample in its own cluster together with its nearest neighbor and then, the clusters that share samples are merged. This results in a number of initial clusters that is at least half the number of samples.

\subsection{Building the structural graph}

We define the structural graph G as the directed graph where each initial sample X is a node and E the set of edges connecting them. Two samples are connected by an edge if they are in its K closest neighbors (for a predefined graph).

Then, we compute the weighted adjacency matrix W, which contains the pairwise similarities between each pair of samples. Note that we are only computing similarities between each point and its K closest neighbors, the rest of similarities will be 0. Therefore, each element \(w_ij)\) of the matrix W is defined as shown in Equation \ref{eq1}.

\begin{equation}
    w_{ij} =
    \begin{cases} 
    \exp \left( \frac{-\text{dist}(i,j)^2}{\sigma^2} \right), & \text{if } \mathbf{x}_j \in \mathcal{N}_i^K, \\
    0, & \text{otherwise}.
    \end{cases}
    \label{eq1}
\end{equation}

Where \textit{K} is a free parameter to be set and \(\sigma^2\) is estimated by Equation \ref{eq4}.

\begin{equation}
    \sigma^2 = [\sum_{j=1}^{n}\sum_{x_j\in \mathcal{N}_i^3}dist(i,j)^2] / [3n(-ln(a))]
    \label{eq4}    
\end{equation}

Where \textit{a} is a parameter to be set.

We define the transition probability matrix \textit{P} as the one-step transition probability from vertex i to vertex j, and we compute it using Equation \ref{eq5}.

\begin{equation}
    \begin{split}
    P = D^{-1}W; 
    \\
    d_{ii} = \sum_{j=1}^n w_{ij}
    \text{ such that }
    \sum_{j=1}^n p_{ij}=1
    \end{split}
    \label{eq5}
\end{equation}

\subsection{Affinity measure: Path integral}

The path integral of a cluster is computed by summing the paths in the cluster on the directed graph \textit{G}, weighted by transition probabilities in \textit{P}.

The path integral used is a discretization of the one seen in quantum mechanics, and is defined as a sum of the weighted contributions of all the paths in the cluster C, divided by the number of samples belonging to the cluster. This is called the path integral descriptor of a cluster \(C_a\), which we call \(S_{C_a}\), and is defined in Equation \ref{eq2}.
\ref{eq2}.

\begin{equation}
    S_c = \frac{1}{|C|^2}\sum_{\gamma\in\Gamma_c}{\Theta(\gamma)}
\label{eq2}
\end{equation}

Where, \(\Gamma_c\) is the set of all paths in C and \(\Theta(\gamma)\) the weight of a path.

We also define the conditional path integral descriptor \(S_{C_a|C_a\cup C_b}\) between two clusters \(S_{C_a}\) and \(S_{C_b}\) as the path descriptor of the clustering resulting from the merge but only taking into account the paths that have starting and ending vertices in \(C_a\). We compute the conditional path integral descriptor using Equation \ref{eq6}.

\begin{equation}
    S_{C_a|C_a\cup C_b} = \frac{1}{|C_a|^2} \boldsymbol{1}_{C_a}^{T}(\boldsymbol{I}-z\boldsymbol{P}_{C_a\cup C_b})^{-1}\boldsymbol{1}_{C_a}
    \label{eq6}
\end{equation}

Now, we can compute the affinity \(A_{C_a,C_b}\) of the merging of two clusters \(C_a\) and \(C_b\) by the increment in the path integral descriptor shown in the previous equations \ref{eq2} and \ref{eq6}, as can be seen in Equation \ref{eq3}. In addition to the path integral of each cluster separately \(S_{C_a}\) and \(S_{C_b}\), the computation of the affinity uses the conditional path integral descriptor \(S_{C_a|C_a\cup C_b}\), which computes the resulting path descriptor of the merge but only the ones that have starting and ending vertices in \(C_a\).

\begin{equation}
    A_{C_a,C_b} = (S_{C_a|C_a\cup C_b}-S_{C_a}) + (S_{C_b|C_a\cup C_b}-S_{C_b})
    \label{eq3}
\end{equation}

\subsection{Agglomerative algorithm}

The input to our problem is a set of sample vectors together with the target number of clusters (that has to be predefined).

The algorithm starts by building the graph, weighted adjacency matrix \(W\), and using it to obtain the transition probability matrix \(P\). Then, cluster initialization is run to obtain the initial clusters.

The iterative part consists of merging the two most affine clusters we have until the number of clusters is the target one. To do so, we use the precomputed affinities between all pairs of clusters. The pseudocode of the algorithm can be seen in Algorithm \ref{algorithmpseudo}.

\begin{algorithm}
\caption{Path Integral Clustering Algorithm}
\begin{algorithmic}[1]
    \State \textbf{INPUT} The set of $n$ points to cluster $\mathcal{X} = \{\mathbf{x}_1, \mathbf{x}_2, \dots, \mathbf{x}_n\}$, and the target number of clusters $n_T$.
    \State Build the graph $G$ with $k$-nearest-neighbors and compute its weighted adjacency matrix $\mathbf{W}$.
    \State Get the transition probability matrix $\mathbf{P}$
    \State Run nearest neighbour merging to form $n_c$ initial clusters $\mathcal{C}_c = \{\mathcal{C}_1, \ldots, \mathcal{C}_{n_c}\}$.
    \While{$n_c > n_T$}
        \State Search two clusters $\mathcal{C}_a$ and $\mathcal{C}_b$, such that 
        \[
        \{\mathcal{C}_a, \mathcal{C}_b\} = \arg\max_{\mathcal{C}_a, \mathcal{C}_b \in \mathcal{C}_c} \mathcal{A}_{\mathcal{C}_a, \mathcal{C}_b}
        \]
        where $\mathcal{A}_{\mathcal{C}_a, \mathcal{C}_b}$ is the affinity measure between $\mathcal{C}_a$ and $\mathcal{C}_b$.
        \State $\mathcal{C}_c \leftarrow \{\mathcal{C}_c \setminus \{\mathcal{C}_a, \mathcal{C}_b\} \} \cup \{\mathcal{C}_a \cup \mathcal{C}_b\}$
        \State $n_c \leftarrow n_c - 1$
    \EndWhile
    \State \textbf{OUTPUT} $\mathcal{C}_c$
\end{algorithmic}
\label{algorithmpseudo}
\end{algorithm}

\section {Description of the implementation}

In this section we are going to describe the particularities of the implementation of the Path Integral Clustering Algorithm. The first section goes over the implementation techniques recommended by the paper, and the second one contains the extra decisions made during the implementation. 

During the implementation, efficiency was our priority. The full implementation of the algorithm is contained in the files \textit{"pic.py"}, \textit{"nearest\_neighbour\_init.py"} and \textit{"path\_integral.py"}. The rest of files are helpers that have contributed to the results obtained but are not part of the implementation itself.

\subsection{Efficient computation of the path integral}

The main ideas laid out by the paper about the implementation are on the computation of the path integral between two clusters. They use Theorem 1 to avoid computing a matrix inverse but instead solve a linear system. 

\textbf{Theorem 1}. $s_{ij}$ always converges, and $s_{ij}= \left[(I - zP_{\mathcal{C}})^{-1}\right]_{ij}$, i.e., the $(i,j)$-element of $(I-zP_{\mathcal{C}})^{-1}$, where $P_{\mathcal{C}}$ is the submatrix of $P$ by selecting the samples in $\mathcal{C}$. If we define $S_{\mathcal{C}} = [s_{ij}]_{i,j \in \mathcal{C}}$, we have $S_{\mathcal{C}} = (I - zP_{\mathcal{C}})^{-1}$. Then, we can compute the path integral as the structural descriptor of cluster $\mathcal{C}$ as follows:

\begin{equation}
    S_{\mathcal{C}} = \frac{1}{|\mathcal{C}|^2} \mathbf{1}^{T} S_{\mathcal{C}} \mathbf{1}
    = \frac{1}{|\mathcal{C}|^2} \mathbf{1}^{T} (I - zP_{\mathcal{C}})^{-1} \mathbf{1}
    \label{eq8}
\end{equation}

where $\mathbf{1}$ is all-one column vector.

\textbf{Proposition 1} $(I - zP_{\mathcal{C}})$ is a strictly diagonally dominant matrix with the $\infty$-norm condition number no more than $(1+z)/(1-z)$.

Both Theorem 1 and Proposition 1 are proven in the original paper \cite{citation1}, so we will not go into more detail here. Taking them into account, we see that the computation of $S_{\mathcal{C}}$ only involves solving a linear system

\begin{equation}
    \begin{split}
    (I - zP_{\mathcal{C}}) \mathbf{y} = \mathbf{1},
    \label{eq7}
    \\
    S_{\mathcal{C}} = \frac{1}{|\mathcal{C}|^2} \mathbf{1}^{T} \mathbf{y}.
    \end{split}
\end{equation}

For a large cluster, $(I - zP_{\mathcal{C}})$ is sparse, which means it follows Proposition 1 and can be solved using iterative methods. This will help us compute both the incremental path integral and the path integral descriptor.

\begin{equation}
    S_{c_a|c_a \cup c_b} = \frac{1}{|c_a|^2} \mathbf{1}_{c_a}^{T} (I - zP_{c_a \cup c_b})^{-1} \mathbf{1}_{c_a}
\end{equation}

To compute the path integral descriptor as shown in Equation \ref{eq8}, we first compute \((I - zP_{c_a \cup c_b})\), which is straightforward. Then, following equation \ref{eq7}, we solve the linear system (we can use spsolve). Finally, multiplying by the transposed matrix of 1s is the same as summing all the elements in the matrix resulting from the linear equation's solving, and finally we can compute the multiplication with the fraction. Computing the incremental path descriptor takes advantage of the same trick. 

\subsection{Implementation decisions}

Because we want our algorithm to run in linear time, we need to precompute everything we can, and that includes not only the graph and its transition matrices, but also the affinity between each pair of clusters. Because as can be seen in \ref{algorithmpseudo} the decision on the clusters to merge is an argmax, we need to have computed the affinities between all pairs of clusters to compare them.

In addition to this, we are using a heap to store all the cluster pairs and their affinities for fast extraction in runtime. This brought us to use a set to store the clusters that are active in a moment, to avoid having to traverse the heap to remove all pairs of clusters that contains one of the merged ones at each iteration. This means that at every iteration we are flagging the merged clusters as inactive, computing the affinities between the new cluster and all the previous ones, and adding them to the heap.

Due to the nature of our algorithm, we only implement the fit-predict method. This is because the algorithm does not learn from data but instead applies the same process to the samples to cluster them independently from the data it has seen.

All the operations that involve the clustering algorithm use numpy, using matrices as numpy arrays, and taking advantage of the different tools we have to work with sparse matrices. Because of the way the similarities are computed, being set to zero if they are not in the K nearest neighbors of every sample, and because of their enormous sizes, we use different tools from scipy.sparse, including spsolve to solve the linear systems that compute the Path Integral (see previous section), converting matrices to csc form and making computations between matrices using the sparse tools.

\section {Experiments}

In this section we will conduct experiments comparing different clustering algorithms on several datasets. The algorithms under evaluation include Affinity Propagation, Complete Link, Average Link, Simple Link, Zeta clustering (Zell), Difussion clustering, and of course, our implementation of PIC. Each algorithm will be tested on four datasets: MNIST, USPS, Breast Cancer (Wisconsin), and a synthetic random dataset with 1000 samples and 50 features. We are going to divide the experiments section into two subsections. In the first subsection, we are going to analyse how does the introduction of gaussian noise and structural noise affect the performance of the different algorithms, similar to what is done in the original paper. In the second one, we are going to run the algorithms on the complete datasets to observe what are the differences in their performances, using a diverse set of metrics.

The choice of the algorithms we are using is due to the fact that we only have found implementations for the Affinity Propagation, Complete, Average and Simple Links. Although in the project we only had to implement PIC, we decided to also implement some of the algorithms that get the most similar results in the paper, and the selected algorithms to implement are the diffusion Kernel (D-kernel) and Zeta function clustering (Zell), because they are shown to be better than PIC in some domains in the original paper.

The datasets chosen are also extracted from the paper: MNIST and USPS. We opted not to use the FRGC ver2.0, PubFig, and Caltech datasets for this study due to their large size, which would require significant computational resources and time for processing, apart from the approach the paper takes on loading them, which uses spatial pyramid features. Instead, we have decided to incorporate the Breast-cancer wisconsin dataset because it is a well-known benchmark in the medical domain, containing a high-dimensional, real-world classification and being widely used. In addition, being a binary classification dataset, it stands out from the rest and means that the algorithms will have to iterate until only two clusters are remaining, apart from being something not tested in the original paper.

We don't have access to the synthetic datasets that are used in the paper, and for this reason we have decided to create our own. It will be using the synthetic datasets that we will study how does structural and gaussian noise affect our implementation of the clustering algorithm. We are going to use three different synthetic datasets, the first one using moons, second one using concentric circles and last one using blobs. Our synthetic datasets will have 1000 samples, 2 features for ease of representation (same as the ones in the paper) and 10 target clusters.

\begin{table}[h]
\centering
\caption{Statistics of used datasets.}
\begin{tabular}{lccccc}
\toprule
\textbf{Dataset} & \textbf{USPS} & \textbf{MNIST} & \textbf{BC-Wisconsin} & \textbf{Synthetic}\\
\midrule
No. of samples     & 11\,000 & 5\,139 & 569 & 1000 \\
No. of clusters    & 10     & 5     & 2    & 10   \\
Min. cluster size  & 1100   & 980   & 212     & - \\
Max. cluster size  & 1100   & 1135  & 357     & - \\
Noise added  & No   & No  & No     & Yes \\
Dimensionality     & 256    & 784   & 30   & 2 \\
\bottomrule
\end{tabular}
\end{table}

\subsection{Metrics to be used}

The metrics we will be using to compare the methods are the same ones that are used in the paper: Normalized Mutual Information score and Clustering error, and as both of these are external metrics we decided to incorporate some internal ones: Silhouette score, Davies-Bouldin score, and Calinski-Harabasz Index. We have selected these three internal metrics because they are the ones recommended as internal criteria for a wide range of situations by the "Clustering evaluation and validation" chapter of the material of the Unsupervised Learning course \cite{citation2}. Their implementations can be seen in Equations \ref{eq9}, \ref{eq10}, \ref{eq11}, \ref{eq12}, and \ref{eq13}.

\begin{equation}
    \text{NMI}(U, V) = \frac{2 \times I(U; V)}{H(U) + H(V)}
    \label{eq9}
\end{equation}
where:
\begin{itemize}
    \item $I(U; V)$ is the mutual information between cluster assignment $U$ and ground truth labels $V$.
    \item $H(U)$ and $H(V)$ are the entropies of $U$ and $V$.
\end{itemize}

\begin{equation}
    \text{CE} = 1 - \frac{1}{N} \sum_{i=1}^{N} \mathbb{1}\{ \hat{y}_i = y_i \}
    \label{eq10}
\end{equation}
where:
\begin{itemize}
    \item $N$ is the total number of samples.
    \item $\hat{y}_i$ is the predicted cluster label after optimal matching.
    \item $y_i$ is the ground truth label.
    \item $\mathbb{1}\{\cdot\}$ is the indicator function (1 if true, 0 otherwise).
\end{itemize}

\begin{equation}
    s(i) = \frac{b(i) - a(i)}{\max\{a(i), b(i)\}}
    \label{eq11}
\end{equation}
where:
\begin{itemize}
    \item $a(i)$ is the average distance from point $i$ to all other points in the same cluster.
    \item $b(i)$ is the minimum average distance from point $i$ to points in a different cluster.
\end{itemize}
The overall Silhouette Score is the mean of $s(i)$ over all points.

\begin{equation}
    \text{DB} = \frac{1}{k} \sum_{i=1}^{k} \max_{j \neq i} \left( \frac{S_i + S_j}{M_{ij}} \right)
    \label{eq12}
\end{equation}
where:
\begin{itemize}
    \item $k$ is the number of clusters.
    \item $S_i$ is the average distance between each point in cluster $i$ and its centroid.
    \item $M_{ij}$ is the distance between the centroids of clusters $i$ and $j$.
\end{itemize}

\begin{equation}
    \text{CH} = \frac{\text{Tr}(B_k)}{\text{Tr}(W_k)} \times \frac{N - k}{k-1}
    \label{eq13}
\end{equation}
where:
\begin{itemize}
    \item $N$ is the total number of points.
    \item $k$ is the number of clusters.
    \item $\text{Tr}(B_k)$ is the trace of the between-cluster dispersion matrix.
    \item $\text{Tr}(W_k)$ is the trace of the within-cluster dispersion matrix.
\end{itemize}

\subsection{Noise analysis}

In this section we will run experiments using the previously described synthetic dataset incorporating different levels of gaussian noise and structural noise to see how do the different methods behave. Because the addition of gaussian and structural noise is done randomly, we have followed the same practices from the original paper and repeated all the experiments 20 times, showing the mean and the standard deviation at each noise level.



\subsection{Algorithm comparison}

This section presents a comparative analysis of several clustering algorithms across four datasets: USPS, MNIST, BC-Wisconsin, and a Synthetic dataset.

\begin{table}[h]
    \centering
    \caption{Quantitative clustering results in NMI. The best values are in bold.}
    \begin{tabular}{lccccc}
    \toprule
    \textbf{Dataset} & \textbf{USPS} & \textbf{MNIST} & \textbf{BC-Wisconsin} & \textbf{Synthetic}\\
    \midrule
    AP       & 0.5176   & 0.4170 & 0.2596  & 1 \\
    A-link   & 0.01128  & 0.0016 & 0.0151  & 1   \\
    S-link   & 0.0019   & 0.0016 & 0.0102  & 1 \\
    C-link   & 0.1592   & 0.0016 & 0.0102  & 1 \\
    Zell     & 0.5743   & \textbf{0.6293} & \textbf{0.5679}  & 1 \\
    D-kernel & 0.2455   & 0.0038 & 0.0102  & 1 \\
    PIC & \textbf{0.8708} & 0.0063 & 0.0102 & 1 \\
    \bottomrule
    \end{tabular}
\end{table}

\begin{table}[h]
    \centering
    \caption{Quantitative clustering results in Clustering error.The best values are in bold.}
    \begin{tabular}{lccccc}
    \toprule
    \textbf{Dataset} & \textbf{USPS} & \textbf{MNIST} & \textbf{BC-Wisconsin} & \textbf{Synthetic}\\
    \midrule
    AP       & 0.9272   & 0.9527 & 0.8963  & 0 \\
    A-link   & 0.8285  & 0.7785 & 0.3673  & 0   \\
    S-link   & 0.8328   & 0.7785 & 0.3691  & 0 \\
    C-link   & 0.7634   & 0.7785 & 0.3691  & 0 \\
    Zell     & 0.4450   & \textbf{0.3592} & \textbf{0.0844}  & 0 \\
    D-kernel & 0.6698   & 0.7772 & 0.3691  & 0 \\
    PIC & \textbf{0.1150} & 0.7772 & 0.3691 & 0 \\
    \bottomrule
    \end{tabular}
\end{table}

The Normalized Mutual Information (NMI) metric quantifies the similarity between clustering assignments and ground truth labels, where higher values indicate better alignment. According to the original paper, the implemented algorithm outperformed other methods on the USPS and MNIST datasets. However, the current implementation and evaluation reveal some notable differences.

On the USPS dataset, PIC indeed achieves the highest NMI value of 0.8708, confirming the original paper's claim and reaffirming its strong performance in capturing the underlying cluster structure. 

For MNIST, a significant divergence from the paper's findings emerges. While PIC is reported as the top performer in the original study, it achieves only 0.0063 NMI here, with Zell dramatically outperforming it at 0.6293. This inversion suggests that in this replication, either PIC struggled to converge to meaningful clusters, or its assumptions were not well-suited to the MNIST data in this particular implementation. It could also imply that Zell, perhaps under different parameter settings or with the specific MNIST preprocessing used here, was better able to exploit the dataset's structure.

On BC-Wisconsin, which was not reported in the original paper, Zell again clearly leads with an NMI of 0.5679, followed by AP at 0.2596. All other methods, including PIC, performed very poorly here, indicating that Zell may be particularly effective in handling this dataset's cluster characteristics — likely benefitting from its capacity for detecting more globally separated clusters, as seen in earlier Silhouette and CH scores.

\begin{table}[h]
    \centering
    \caption{Quantitative clustering results in Silhouette score}
    \begin{tabular}{lccccc}
    \toprule
    \textbf{Dataset} & \textbf{USPS} & \textbf{MNIST} & \textbf{BC-Wisconsin} & \textbf{Synthetic}\\
    \midrule
    AP       & 0.077   & 0.0558 & 0.0870  & 0 \\
    A-link   & 0.2806  & 0.7832 & 0.6339  & 0   \\
    S-link   & 0.3089   & 0.7832 & 0.6606  & 0 \\
    C-link   & 0.1945   & 0.7832 & 0.6606  & 0 \\
    Zell     & 0.0681   & -0.0389 & 0.3263  & 0 \\
    D-kernel & 0.6698   & 0.6349 & 0.6606  & 0 \\
    PIC      & 0.1390  & 0.5941 & 0.6606 & 0 \\
    \bottomrule
    \end{tabular}
\end{table}

\begin{table}[h]
    \centering
    \caption{Quantitative clustering results in DBI}
    \begin{tabular}{lccccc}
    \toprule
    \textbf{Dataset} & \textbf{USPS} & \textbf{MNIST} & \textbf{BC-Wisconsin} & \textbf{Synthetic}\\
    \midrule
    AP       & 1.8701   & 1.5212 & 1.3622  & 0 \\
    A-link   & 1.1553  & 0.1325 & 0.6798  & 0   \\
    S-link   & 0.4623   & 0.1325 & 0.4497  & 0 \\
    C-link   & 2.4111   & 0.1325 & 0.4497  & 0 \\
    Zell     & 2.0193   & 3.3001 & 1.3537  & 0 \\
    D-kernel & 2.1858   & 3.0414 & 0.4497  & 0 \\
    PIC     & 2.3121    & 1.5937 & 0.4497 & 0 \\
    \bottomrule
    \end{tabular}
\end{table}

\begin{table}[h]
    \centering
    \caption{Quantitative clustering results in Calinski-Harabasz}
    \begin{tabular}{lccccc}
    \toprule
    \textbf{Dataset} & \textbf{USPS} & \textbf{MNIST} & \textbf{BC-Wisconsin} & \textbf{Synthetic}\\
    \midrule
    AP       & 68.3675   & 25.6472 & 39.0207  & 0 \\
    A-link   & 22.2013  & 52.8586 & 32.4822  & 0   \\
    S-link   & 4.5830   & 52.8586 & 27.8726  & 0 \\
    C-link   & 242.2214   & 52.8586 & 27.8726  & 0 \\
    Zell     & 697.2199   & 185.1767 & 258.9817  & 0 \\
    D-kernel & 337.163   & 46.1865 & 27.8726  & 0 \\
    PIC      & 517.6182 & 27.6675 & 27.8726 & 0 \\
    \bottomrule
    \end{tabular}
\end{table}

The Silhouette Score, which measures how similar an object is to its own cluster compared to other clusters, revealed that on the MNIST dataset, the A-link, S-link, and C-link methods all achieved identical and notably high scores of 0.7832. This indicates that these methods produced clusters with a high degree of compactness and separation. On the USPS dataset, the D-kernel method outperformed others with a Silhouette Score of 0.6698, suggesting it formed well-defined clusters in this setting. For the BC-Wisconsin dataset, the S-link, D-kernel, and PIC methods all tied with a strong score of 0.6606.

When considering the Davies-Bouldin Index (DBI), where lower values signify better clustering, consistent patterns are found. On the USPS dataset the S-link method achieves the lowest DBI at 0.4623, outperforming other algorithms in terms of minimizing intra-cluster variance while maximizing inter-cluster separation. For the BC-Wisconsin dataset, the S-link, C-link, D-kernel, and PIC methods shared a DBI value of 0.4497, proving their effectiveness. Zell, D-kernel, and PIC generally performed worse in this metric on MNIST and USPS, despite competitive performances elsewhere.

The Calinski-Harabasz Index (CHI), which rewards high between-cluster dispersion and low within-cluster dispersion, offered a contrasting perspective. The Zell method achieved exceptionally high CHI scores on all three real datasets, particularly on BC-Wisconsin, where it recorded a value of 258.98 — considerably surpassing other algorithms. On the USPS dataset, Zell again led with a CHI of 697.22, followed by PIC and D-kernel with notable but lower scores. Although Zell consistently achieved high CHI values, it did not perform as well in the Silhouette and DBI metrics, suggesting that while it can create globally dispersed clusters, they might lack local compactness.

In summary, the results highlight that S-link and A-link methods exhibit balanced, reliable performance across all datasets and metrics, making them strong general-purpose clustering techniques. The D-kernel method excels particularly on the USPS dataset, while Zell demonstrates outstanding performance in terms of global cluster dispersion (as measured by CHI) but falls behind in terms of compactness and separation as captured by the Silhouette and DBI scores. The Synthetic dataset’s consistent zero scores across all methods suggest either a lack of inherent cluster structure or a potential oversight in the evaluation process that warrants further investigation.

These findings underscore the importance of using multiple complementary metrics to assess clustering performance, as different algorithms may excel under different criteria. Based on this analysis, S-link and A-link emerge as particularly reliable choices when seeking balanced clustering performance across various datasets.

\section{Conclusions}

Comment about the claims of the paper, reproducibility \& issues encountered.

\newpage
\section{Bibliography}

\renewcommand{\section}[2]{}%
\begin{thebibliography}{9}

\bibitem{citation1}
Wei Zhang et al.,
\textit{Agglomerative clustering via maximum incremental path integral}, 2013
\href{https://www.sciencedirect.com/science/article/pii/S0031320313001830}{(Link)}

\bibitem{citation2}
Javier Béjar,
URL - 2025 Spring Term
\textit{Clustering evaluation and validation}

\end{thebibliography}

\end{document}